\documentclass[10pt]{article}

\begin{document}
\section*{Individual Report 3}
Donald Aingworth\\
SDG3 Teammates: Beatrice Aragones, Kaitlin Jeon, and Alena Pushniakova\\
Written in \LaTeX. Please excuse any gramatical errors or typos. 

\textbf{Provide your full agenda for your team's Milestone 2 meeting. Include the names of all team members that attended the meeting. Did you have any follow-up meetings? If so, who attended those additional meetings?}\\
There were two meetings since our last report. 
The first was February 25. It focused mostly on UML drafting. In that meeting, the following issues were addressed.
\begin{itemize}
    \item Frontend-Backend Internal Interaction
    \item Frontend-Backend Concurrent Interaction
    \item User-product-machine Interaction
\end{itemize}
Subsequant that meeting, we did make designs for the project. 
Those of us in frontend worked on designs for how the final project would look. Meanwhile, those of us in backend worked on creating class-based UML designs for the overall project. Since the overall project is going to be in Java, we took advantage of the language's features like interfaces. Our focus was on class-based and backend-based design of the project's features. This led into some of the points discussed in the second meeting.

After several delays due to a lack of time on the part of some team members, the second meeting was held on March 18. In that meeting, the following issues were addressed.
\begin{itemize}
    \item Should we use MySQL or a server? Maybe SQLite?
    \item Admin/Account page, where user can see all assignments and general data about them.
    \item Option to edit and delete homework assignemtns from the Admin/Account page.
    \item Push notifications for upcoming assignments.
    \item Labels for classes
    \item Sidebar navigation buttons
    \item Use of ``preferred study times'' and ``email'' tools
\end{itemize}
These largely served to clarify that which we did in our first meeting. 
At this point going forward, we will likely use the foundation from the second meeting to make our prototype. We have been exploring the relationships and cause-effect workings of frontend and backend.

\textbf{How did your team divide the work necessary to complete the Milestone 2 assignment?  What did your team do well and what improvements would you have liked to see in your team's Milestone 2 submission?}\\
We divided our work for the Milestone 2 assignment similarly to how we divided it up for the Milestone 1 Assignment. We made a list of things we should have covered in the milestone, then we each took our pick of what each of us would do. After that, we all looked over each others' and added anything that was missing. Since our team is cleanly divided between frontend and backend, there was a lot missing from others' sections.
Next time, I would hope to do more in the report. I had a lot going on with other classes, so I did not have the time to work on much of the milestone assignment. I hope to get more work done for the Milestone 3 Assignment.

\textbf{Describe one thing that you learned from your experience with one or more of the following: }
\begin{itemize}
    \item Agile Process Model Artifacts: I learned that use of the agile process model and the rate at which processes are completed increases exponentially over time. Early assignments are very big and vague, but they give rise to more and more specific and small assignments. UML helps with this because it sets out all the things that would need to be completed so the user can complete them all individually. More tasks will be in the pipeline in the to-do column, but their turnover rate from To-Do through In-Progress to Done is much faster.
    \item Component-Level Design: Most of the Object-Oriented component-level design was simple and already known. However, doing this was good review for me and it gave me experience with working on object-oriented UML, including re-learning to identify components of more abstract ideas and tasks.
\end{itemize}

\end{document}