\documentclass[12pt]{article}

\title{Individual Report 1}
\author{Donald Aingworth}
\date{February 10, 2025}

\begin{document}
% \maketitle 
\section*{Individual Report 1}
Donald Aingworth

\textbf{Which project did you select and who is on your team?}\\
My group's project, as of yet, is called `Homework Tracker'. 
The project consists of both a calendar and a timer. 
The calendar contains assignments and due dates for classes, which the user can create, delete, move, or otherwise modify.
The timer can be run by the user. One idea for it was that it could have flipdown animation.

I have two teammates: Beatrice and Kaitlin. Beatrice is the one who came up with the idea of the project, while Kaitlin is also quite invested in the project.\par

\textbf{You will need to coordinate with your teammates to decide on an initial meeting.  What is the date, time and format (in-person, zoom, or similar) of your initial meeting?}\\
Our entire group has been in contact on Discord. We have coordinated on that platform for our first meeting. Our meetings will (likely) be weekly on Discord at 8pm on Tuesdays. I have significant experience in working on projects and clubs on Discord, which I believe can be applied to this.


\textbf{List at least three (3) agenda items that you would like to see on the agenda for your team's initial meeting.}\\
1. Do we reasonably think we can make a working prototype of the project in the given time frame? I know that making a working prototype is not explicitly required for the overall project, but I would like to know if we will be working on it or if we will be merely working on it as a theoretical project.\\
2. What process model are we going to use in this project?\\
3. How are we going to integrate the two parts of our project (the calendar and the timer) into one another? Should we make the timer automatically set to the time remaining for a selected task? if none, what is the point of keeping the two of the ideas as one project and not merely doing two separate projects?


\textbf{Compare and contrast two (2) of the process models and evaluate which process model would be the approach that you most prefer for the project you have selected.  Be sure to give detailed reasons why it fits your specific project.}\\
After unilaterally ruling out Scrum due to the nature of the overall project, I will compare Kanban and eXtreme Programming.
Kanban consists of flowing one of multiple tasks from a backlog, through a process of analysis, development, and testing through to completion, collecting information about it while working through it. Its benefits include contnuous improvement and early project delivery and updating, along with specified roles. However, it requires good team collaboration amd a requirement of larger levels of focus.

Meanwhile, eXtreme Programming (XP) is a process that involves heavy and continuous communication with the customer(s). XP requires an extreme (hence the name) amount of communication and collaboration with a customer. It relies heavly on user stories from its start to its end. It also requires frequent criterion set for an acceptance test and frequent modifications of these tests.

As a whole, there are risks to both Kanban and XP. If we go with Kanban, we are running the risk of low unction if our team does not end up being able to communicate well with one another. On the other hand, if we don't find a committed customer, then XP will be anywhere from heavly difficult to outright impossible. I would personally give the edge to KanBan, because I am more confident n my ability ot work with my groupmates than my ability ot find a customer fast.


\textbf{Prepare a list of at least five (5) questions that you might ask the client in order to come up with requirements, user stories or tasks that are in-line with the process model you prefer for your project.}\\
1. How many due dates do you tend to have per day and/or per week?\\
2. About how long is your attention span for a project you are invested in? Not very invested in?\\
3. How do you estimate how long your projects take to complete?\\
4. What difficulties do you have with connection of timers and of calendars (i.e. the clock and the calendar)?\\
5. About how many projects do you have to complete at once?

\end{document}