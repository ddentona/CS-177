\documentclass[10pt]{article}

\title{Individual Report 1}
\author{Donald Aingworth}
\date{February 10, 2025}

\begin{document}
% \maketitle 
\section*{Individual Report 2}
Donald Aingworth\\
Written in \LaTeX. Please excuse any gramatical errors or typos. 

\textbf{Provide your full agenda for your team's initial meeting. Include the names of all team members that attended the meeting. Did you have any follow-up meetings? If so, who attended those additional meetings?}\\
We had two meetings. In our first meeting, we had the following items as the agenda.
\begin{itemize}
    \item Finalizing (major) features
    \item Techstack to use
    \item Process model(s)
    \item Can we get this done in the given time frame?
    \item Team name and roles
    \item Task List (Incomplete)
    \item Risk Analysis
    \item Communication
\end{itemize}
In that meeting, everyone who was in the group at the time was present at least part of the time. Beatrice and I were there the whole time, while Katilin arrived late.\\
In the second meeting, one week later, we had the following items on the agenda.
\begin{itemize}
    \item Roles (Redux)
    \item Web App or Software Application?
    \item Individually written questions
    \item Notes about using APIs
\end{itemize}
It was a full-team meeting. Everyone was present and timely, including out ``new recruit" Alena.

\textbf{Which process model (e.g. Waterfall, Scrum, etc)  did your team select? Why did your team choose that process model for your specific project? Is the model chosen by your team different from the one you prefer? How did you come to a consensus as a team on the process model?}\\
Our team's process model was quickly decided to be Agile through Kanban. We generally agreed on this from the start. The consensus was our general consensus from the start. 

\textbf{Did you have any assumptions about the product that were validated or debunked by the client interviews?}\\
Most of my assumptions about the product were validated by the client interviews. One thing I was surprised about from the client interview was the idea of adding break time suggestions for users. I do think we could implement these.

\textbf{Now that you have had some interaction with your team, how confident are you that the team will work well together? Do you foresee any risks to team success?  What is your personal plan for mitigating risks?}\\
I am somewhat confident that our group can work fine if not well together. The issue I can think of is a lack of communication among the team. We have tried to mitigate this risk by selecting the appropriate model of Kanban, which would allow us to work separately on a single project.
Another risk we have is the question of how we can sort out who can and will do what, in addition to how we operate without others' programs completely finished. The answer was found in this week's discussion, which was about the content of an article about group work. The article from the DBQ argued thae component-centric workflow is useful, including focused design, which we do (or at least should) intend to use here. We additionally have here decided to use the language Java because of (among other things) its interfaces, classes, etc. and because it operates the same on every machine it runs on.

\textbf{How did your team divide the work necessary to complete the Milestone 1 assignment?  What did your team do well and what improvements would you have liked to see in your team's Milestone 1 submission?}\\
We did quite well with the Milestone 1 Assignment. We divided the sections roughly evenly amongst ourselves. We had our sections and claimed them by doing them, in a sort of first-come-first-served strategy. The only critique I might add is that we could have spent a bit of time from our meetings working on it. That is not the biggest issue, though.

\textbf{Describe one thing that you learned from your experience with requirements engineering.}\\
One thing I learned about requirements engineering is that what the client wants and what the team thinks they want can align together quite well, although it can also differ in small but specific ways. Otherwise, what we went through was nothing that I considered to be especially out of the ordinary.


\end{document}