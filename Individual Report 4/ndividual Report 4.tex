\documentclass[10pt]{article}

\begin{document}
\section*{Individual Report 4}
Donald Aingworth\\
SDG3 Teammates: Beatrice Aragones, Kaitlin Jeon, and Alena Pushniakova\\
Written in \LaTeX. Please excuse any gramatical errors or typos. 

\textbf{Provide your full agenda for your team's initial meeting. Include the names of all team members that attended the meeting. Did you have any follow-up meetings? If so, who attended those additional meetings?}\\
For once, our team has only had one meeting between reports. 
Below is the agenda for the meeting, including the topics discussed and solutions found.
\begin{itemize}
    \item Real code vs. prototype? Will use prototype instead
    \item Project Milestone 3 Discussion
    \begin{itemize}
        \item UX Design discussion (Beatrice \& Kaitlin)
        \item Figma screenshots
        \item Risk management (identify risks and mitigation strategies) (Alena \& Donald)
        \item Quality plan (Alena)
        \item Security plan (abuse cases, security reqs) (Donald)
    \end{itemize}
\end{itemize}
While the conclusions of all issues are listed above, I did personally advocate for the possibility of making a fully-functional programmed prototype.
As the situation stands, the possibility of that happening seems rather low, but I try do stay hopeful that I can get myself to do something in favor of that goal.

\textbf{How did you come up with the risks for the risk assessment?}\\
For a prior assignment (a discussion), there was an option to answer a question that was very similar to coming up with risk assessment and management plans.
I took a lot of inspiration from my answer to that discussion question when writing the risk management plan.
I largely used the risks I had discovered when writing my response to that discussion question and wrote the solutions to those potential issues.
I also thought up a few potential issues that I had previously not thought of and wrote both them and their potential solutions down.

\textbf{How did your team determine quality standards?}\\
Quality standards was not in my primary line of tasks, but I have an idea of how it was done.
To my understanding, a short list of requirements was made for the quality plan.
Answers to these requirements were then listed by Alena as she answered each of the requirements based on the information we had previously discussed in meetings.

\pagebreak
\textbf{How did your team divide the work necessary to complete the Milestone 3 assignment?  What did your team do well and what improvements would you have liked to see in your team's Milestone 3 submission?}\\
Work on the Milestone 3 Assignment was largely divided up by person based on area of expertise. 
Since the idea of backend implementation as a program has largely been reducing itself to merely an idea and not a reality, those of us previously in backend have been relegated more so to work on issues with documentation and have been working more theoretically.
I think the team did well. 
I especially would lke to commend all my teammates, I think they went quite above and beyond.
One thing I would say is that we started working on this relatively late, so I would like to see us start work on it earlier in the future.

\textbf{Describe one thing that you learned from your experience with UX design.}
I learned about the many pillars of good UX design. 
Most of this came through our quizzes and discussion posts. 
Previously, I had understood UX at its base but had nt understood the components to it, such as the pillars of UX design and the use of use cases.
This information about UX and the criteria thereof make it a lot easier to understand and apply good UX design.

\end{document}