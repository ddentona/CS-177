\documentclass[12pt]{article}

\title{Assignment 1}
\author{Donald Aingworth}
\date{January 27, 2025}

\begin{document}

\maketitle

% \pagebreak
\subsubsection*{3.2. Describe agility (for software projects) in your own words.}
In software engineering, agility is the ability to adapt to change in the process of development. This includes adaptation to improving and changing technology. It focuses on larger amounts of planning and deemphasizes quick procurement of deliverables. Large amounts of planning encompass it.

\subsubsection*{3.5. Why do requirements change so much? After all, don't people know what they want?}
It is true that people know (to a certain extent) what they want in the moment. However, as time passes, different technologies are announced and created. Many people would be willing to wait extra time for a product in return for greater breadth of functionality. For example, Nintendo's \textit{The Legend of Zelda: Breath of the Wild} (2017) was initially intended to be released for the Wii U. However, due to a series of delays and an anouncement of a new console, the video game was also \texttimes released on the next generation console, the Nintendo Switch. In spite of a delay of three years, the game still sold thirty-three million copies and won The Game Award for Game of the Year the year it was released.

\subsubsection*{3.7. Write a user story that describes the "favorite places" or "favorites" feature available on most web browsers.}
As an employee of a large company working regularly in an office and who regularly visits a short list of a few websites, I want an in-browser tool to automatically go to a few commonly-used websites I choose so that I can spend less time getting to these websites and be more productive sooner.

\subsubsection*{4.2. Write the acceptance criteria for the user story that describe the use of the "favorite places" or "favorites" feature found on most web browsers that you wrote for problem 3.7 in Chapter 3.}
The user should be able to add and remove specific websites to a list of links they can access within the browser with only a couple clicks and which open the link in the browser when clicked.

\subsubsection*{4.3. How would you create a preliminary architectural design for the first prototype for a mobile app that lets you create and save a shopping list on your device.}
The list itself would be located in the center. It would be in communication with the hardware/operating system in order to save the contents of the shopping list. Data would flow through one pipeline to get to the OS and another to get to the program. The program would have the ability to add items, modify items, and remove items, as well as the ability to save a new list.

\end{document}